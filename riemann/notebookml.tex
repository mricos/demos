\documentclass{article}
\usepackage{amsmath}
\usepackage{amssymb}
\usepackage{amsfonts}
\usepackage{url}

\begin{document}

\section*{The Connection Between the Riemann Hypothesis and the Error Bound of the Prime Number Theorem}

The query seeks to understand the relationship between two statements: the Riemann Hypothesis (RH) concerning the zeros of the Riemann zeta function (\(\zeta(s)\)), and a specific error bound related to the Prime Number Theorem (PNT). It also addresses the common, but inaccurate, claim that the RH and this error bound are equivalent.

\begin{itemize}
    \item \textbf{Statement 1: \( \zeta(s) = 0 \)} This statement refers to the **Riemann Hypothesis (RH)**, a conjecture about the non-trivial zeros of the Riemann zeta function, \( \zeta(s) \). The RH states that all non-trivial zeros of  \( \zeta(s) \) have a real part equal to \(\frac{1}{2}\). These zeros are complex numbers of the form \( s = \frac{1}{2} + it \), where \( t \) is a real number.  The function  \( \zeta(s) \) is defined for all complex numbers \(s\), except \(s=1\) [2].

    \item \textbf{Statement 2:}  
    {\displaystyle |\pi (x)-\operatorname {li} (x)|<{\frac {1}{8\pi }}{\sqrt {x}}\log(x),\qquad {\text{for all }}x\geq 2657,} This inequality provides a precise bound on the error term in the PNT. It states that the absolute value of the difference between \(\pi(x)\), the prime-counting function, and  \(\text{li}(x)\), the logarithmic integral function, is less than  \(\frac{1}{8\pi} \sqrt{x} \log(x)\) for all \(x \geq 2657\) [3]. This is a refined version of the PNT.

    \item \textbf{The Prime Number Theorem (PNT)}: The PNT gives an asymptotic estimate for the distribution of prime numbers. It states that \(\pi(x) \approx \text{li}(x)\) for large \(x\). However, it does not provide a precise bound on the error term \( \pi(x) - \text{li}(x) \).

   \item  \textbf{Implication, Not Equivalence}: It is crucial to understand that the RH (Statement 1) and the error bound (Statement 2) are \textbf{not equivalent}. The RH is a conjecture that, \textbf{if true, implies the error bound in Statement 2}. In other words, the error bound is a *consequence* of the RH [3]. Von Koch (1901) proved that if the Riemann hypothesis is true, then the error term in the prime number theorem can be bounded by \( O(\sqrt{x} \log x) \). Schoenfeld (1976) refined this result, specifying the bound \( |\pi(x) - \text{li}(x)| < \frac{1}{8\pi} \sqrt{x} \log(x) \) for \( x \geq 2657 \) [3].
    
     \item  \textbf{Why the Connection?}: The connection arises from the fact that the non-trivial zeros of the Riemann zeta function influence the distribution of prime numbers. Riemann's explicit formula shows that the zeros of \( \zeta(s) \) control the oscillations of primes around their expected positions.  The real parts of the non-trivial zeros determine the magnitude of these oscillations [4]. If the zeros all lie on the critical line (\(Re(s) = \frac{1}{2}\)), the error term in the PNT becomes minimal and the distribution of primes is as regular as possible.
    
    \item \textbf{The Constant \(\frac{1}{8\pi}\)}: The specific constant \( \frac{1}{8\pi} \) in the error bound is not arbitrary. It arises from detailed analytic arguments in number theory, including Fourier analysis and explicit formulas that bound the error term [4, 5]. It's believed this constant emerges from calculations involving the non-trivial zeros of the Riemann zeta function, particularly those on the critical line [5]. The constant reflects the oscillatory behavior of the primes.
   
    \item \textbf{The term \(\sqrt{x}\)}:  The appearance of \( \sqrt{x} \) in the error bound is because the real part of the zeros are conjectured to be 1/2 [6].

   \item \textbf{The Logarithmic Term}: The logarithmic term,  \(\log x\), in the bound is related to the density of zeros along the critical line [5].

   \item \textbf{Implications}:  If the RH is true, then the error bound in statement 2 provides a measure of how accurately \( \text{li}(x) \) approximates \( \pi(x) \). If the RH is true, the primes will be distributed in a well understood way with the "best possible" error term [3].
    
\end{itemize}

In summary, while the Riemann Hypothesis is not equivalent to the statement of the error bound, the truth of the RH \textbf{implies} the error bound.  The locations of the non-trivial zeros of the Riemann zeta function are deeply connected to the distribution of prime numbers. If all non-trivial zeros have a real part of 1/2 (as the RH posits), then the error in the prime number theorem is minimized, leading to the specific error bound given in statement 2. The constant \( \frac{1}{8\pi} \) is a consequence of the analytic properties of the zeta function when the Riemann Hypothesis is assumed to be true. The provided link \url{https://www.jstor.org/stable/2005976} contains Schoenfeld's work that gives a precise statement of this bound conditional on RH.
\end{document}