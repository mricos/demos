
\documentclass{article}
\usepackage{amsmath}
\usepackage{amsfonts}
\usepackage{amssymb}

\begin{document}

\section*{The Connection Between the Riemann Hypothesis and the Error Bound of the Prime Number Theorem}

The statement that the Riemann Hypothesis (RH) implies the "best possible" bound for the error of the prime number theorem, specifically 
\begin{equation*}
    |\pi(x) - \text{li}(x)| < \frac{1}{8\pi} \sqrt{x} \log(x),
\end{equation*}
for all $x \geq 2657$, is a significant result in analytic number theory [1]. This error bound, a refinement of the Prime Number Theorem (PNT), was established by von Koch (1901) and made more precise by Schoenfeld (1976). Here is a breakdown of the key ideas connecting the RH to this bound, drawing from the sources and our conversation:

\begin{itemize}
    \item \textbf{The Prime Number Theorem (PNT)}: The PNT provides an asymptotic estimate for the distribution of prime numbers, stating that $\pi(x) \approx \text{li}(x)$, where $\pi(x)$ is the prime-counting function and $\text{li}(x)$ is the logarithmic integral function. The PNT, however, does not give a precise bound on the error term, which is the difference $\pi(x) - \text{li}(x)$.
    \item \textbf{The Riemann Hypothesis (RH)}: The RH is a conjecture about the location of the non-trivial zeros of the Riemann zeta function, $\zeta(s)$. It states that all such zeros have a real part equal to 1/2, i.e., they lie on the critical line $\text{Re}(s) = \frac{1}{2}$ [2]. The location of these zeros is intimately related to the distribution of prime numbers.
    \item \textbf{Error Bound and the Constant}: The error bound $|\pi(x) - \text{li}(x)| < C \sqrt{x} \log(x)$ provides a way to quantify the difference between $\pi(x)$ and $\text{li}(x)$. The constant $C$ is critical in determining the tightness of the bound. The specific constant $C = \frac{1}{8\pi}$ is not arbitrary but arises from deep connections to the behavior of the Riemann zeta function's zeros [3]. The constant $C$ is influenced by the distribution of the zeros of $\zeta(s)$ [4]. If all non-trivial zeros lie on the critical line $\text{Re}(s) = \frac{1}{2}$, the error bound is optimal [5].
      \item The constant $\frac{1}{8\pi}$ may be derived from detailed analytic number theory arguments, particularly when bounding the error term using Fourier analysis or explicit formulas [6].
    \item  \textbf{Von Koch's Result}: Von Koch (1901) proved that if the Riemann Hypothesis is true, then the error term in the prime number theorem can be bounded by $O(\sqrt{x}\log x)$. This means the error $|\pi(x) - \text{li}(x)|$ grows no faster than a constant times $\sqrt{x}\log x$, establishing an explicit link between the RH and the error in the PNT.
    \item \textbf{Schoenfeld's Refinement}: Schoenfeld (1976) provided a precise version of von Koch's result, showing that under the assumption of the Riemann Hypothesis, the error in the prime number theorem can be explicitly bounded by 
      $$
      |\pi(x) - \text{li}(x)| < \frac{1}{8\pi} \sqrt{x} \log(x),
      $$
      for all $x \geq 2657$ [1]. This gave a specific constant and a practical range for $x$ for which the bound applies.
     \item \textbf{Connection to Non-Trivial Zeros}: The constant $\frac{1}{8\pi}$ is thought to originate from detailed calculations with the non-trivial zeros $\rho$ of $\zeta(s)$, particularly those on the critical line which have the form $\rho = \frac{1}{2} + i\gamma$ where $\gamma$ is the imaginary part [5]. The oscillatory contributions of these zeros are believed to cancel out in a way that produces this specific scaling factor [5]. The term $\sqrt{x}$ in the bound arises because the real part of the zeros are conjectured to be 1/2. The logarithmic term comes from the density of zeros along the critical line [6].
     \item \textbf{Implication of the Bound}: The bound given by Schoenfeld shows how well we can approximate $\pi(x)$ using $\text{li}(x)$, given the truth of the RH. If the RH is true, we know that $\pi(x)$ does not deviate from $\text{li}(x)$ too much. The constant $\frac{1}{8\pi}$ suggests that the error bound is tight, reflecting the precise way the zeros of the Riemann zeta function impact the distribution of primes. The bound is considered "best possible" because, given the assumption of the Riemann Hypothesis, it provides the smallest possible growth rate for the error term [7].
     \item \textbf{Analytic Number Theory and Fourier Analysis}: The constant $\frac{1}{8\pi}$ arises in the context of bounding error terms using analytical techniques such as Fourier analysis and explicit formulas [6]. Deriving these bounds often involves integrals with $\ln x$, $\zeta(s)$, and trigonometric functions of the zeros [6].
     \item \textbf{Numerical Verification}: The bound is verified numerically for certain ranges of $x$. For example, computations using the zeros of the zeta function show that the bound is valid for $2657 \leq x \leq 1.101 \times 10^{26}$ [8].
\end{itemize}

In summary, Von Koch's result (refined by Schoenfeld), stating that 
$$|\pi(x) - \text{li}(x)| < \frac{1}{8\pi} \sqrt{x} \log(x)$$ 
if the Riemann Hypothesis is true, establishes a precise relationship between the distribution of prime numbers and the zeros of the Riemann zeta function. The specific constant $\frac{1}{8\pi}$ is a consequence of deep analytic properties of the zeta function and its zeros, which are connected to the oscillatory behavior of primes.
\end{document}